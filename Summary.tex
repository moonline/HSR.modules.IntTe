%Pakete;
%A4, Report, 12pt
\documentclass[ngerman,a4paper,12pt]{scrreprt}
\usepackage[a4paper, right=20mm, left=20mm,top=30mm, bottom=30mm, marginparsep=5mm, marginparwidth=5mm, headheight=7mm, headsep=15mm,footskip=15mm]{geometry}

%Papierausrichtungen
\usepackage{pdflscape}
\usepackage{lscape}

%Deutsche Umlaute, Schriftart, Deutsche Bezeichnungen
\usepackage[utf8]{inputenc}
\usepackage[T1]{fontenc}
\usepackage[ngerman]{babel}

%quellcode
\usepackage{listings}

%tabellen
\usepackage{tabularx}

%listen und aufzählungen
\usepackage{paralist}

%farben
\usepackage[svgnames,table,hyperref]{xcolor}

%symbole
\usepackage{latexsym,textcomp}
\usepackage{amssymb}

%font
\usepackage{helvet}
\renewcommand{\familydefault}{\sfdefault}

%durch- und unterstreichen
\usepackage{ulem}

%Abkürzungsverzeichnisse
\usepackage[printonlyused]{acronym}

%Bilder
\usepackage{graphicx} %Bilder
\usepackage{float}	  %"Floating" Objects, Bilder, Tabellen...
\usepackage[space]{grffile} %Leerzechen Problem bei includegraphics
\usepackage{wallpaper} %Seitenhintergrund setzen
\usepackage{transparent} %Transparenz

%Tikz, Mindmaps, Trees
\usepackage{tikz}
\usetikzlibrary{mindmap,trees}
\usepackage{verbatim}

%for
\usepackage{forloop}
\usepackage{ifthen}

%Dokumenteigenschaften
\title{Summary IntTe}
\author{Tobias Blaser}
\date{\today{}, Uster}


%Kopf- /Fusszeile
\usepackage{fancyhdr}
\usepackage{lastpage}

\pagestyle{fancy}
	\fancyhf{} %alle Kopf- und Fußzeilenfelder bereinigen
	\renewcommand{\headrulewidth}{0pt} %obere Trennlinie
	\fancyfoot[L]{\jobname} %Fusszeile links
	\fancyfoot[C]{Seite \thepage/\pageref{LastPage}} %Fusszeile mitte
	\fancyfoot[R]{\today{}} %Fusszeile rechts
	\renewcommand{\footrulewidth}{0.4pt} %untere Trennlinie

%Kopf-/ Fusszeile auf chapter page
\fancypagestyle{plain} {
	\fancyhf{} %alle Kopf- und Fußzeilenfelder bereinigen
	\renewcommand{\headrulewidth}{0pt} %obere Trennlinie
	\fancyfoot[L]{\jobname} %Fusszeile links
	\fancyfoot[C]{Seite \thepage/\pageref{LastPage}} %Fusszeile mitte
	\fancyfoot[R]{\today{}} %Fusszeile rechts
	\renewcommand{\footrulewidth}{0.4pt} %untere Trennlinie
}

\usepackage{changepage}

% Abkürzungen für Kapitel, Titel und Listen
\input{toolsAndCommands/shortcutsListAndChapter}
\input{toolsAndCommands/TextStructuringBoxes}

%links, verlinktes Inhaltsverzeichnis, PDF Inhaltsverzeichnis
\usepackage[bookmarks=true,
bookmarksopen=true,
bookmarksnumbered=true,
breaklinks=true,
colorlinks=true,
linkcolor=black,
anchorcolor=black,
citecolor=black,
filecolor=black,
menucolor=black,
pagecolor=black,
urlcolor=black
]{hyperref} % Paket muss unbedingt als letzes eingebunden werden!

\usepackage{graphicx}
\begin{document}

% Inhaltsverzeichnis
\tableofcontents
\clearpage


\ch{Java Webtechnologien}
	\stdImg{v1.1}{Web App Request Handling}
	\expl{Servlet Container (Web Container)}{Stellt Services wie Request Dispatching, Livecycle Management, Security, ... zur Verfügung.
	}
	\stdImg{v1.2}{Web Module}
	\expl{Web Archive Adressierung}{http://host:port/app/url-pattern}
	
	
	\se{Servlets}
		\sse{Schnittstellen}
			\stdImg{v1.4}{javax.servlet}
			\stdImg{v1.5}{javax.servlet.http}
			
		\sse{Konfiguration}
			\stdImg{v1.6}{Konfiguration: web.xml}
			
		\sse{Request}
			\stdImg{v1.3}{Request URL}
			\stdImg{v1.7}{Aufruf Mechanismus}
			\stdImg{v1.8}{HttpServletRequest}
			\stdImg{v1.9}{Datenzugriff in Rohform}
			
		\sse{Exception Handling}
			\stdImg{v1.10}{Exception Handling über Error Code oder Exception type (individuelle Error page)}
		
		\sse{Servlet Livecycle}
			\stdImg{v1.11}{Servlet Live Cycle}
			\stdImg{v1.12}{Live Cycle}
			
			\expl{Instanzvariablen}{Müssen Thread save sein, weil das Servlet multithreaded programmiert werden könnte und damit parallele Request abarbeitung möglich ist. Besser: Instanzen im Application Scope speichern, wo alle Servlets zugreifen können.}
			
			\expl{destroy()}{Destroy so implementieren, das es einen Flag setzt. In der Service Methode beende meinen Betrieb ordentlich und so schnell wie möglich. Nach bestimmter Zeit wird das Servlet abgeschossen werden.}
		
		\sse{Scopes}
			\stdImg{v1.13}{Servlet Context}
			
			
		\sse{Session Tracking}
			\stdImg{v1.14}{Grundlegendes zu Sessions}
			\stdImg{v1.15}{Aufbau einer Session}
			\stdImg{v1.16}{Session beenden}
			\stdImg{v1.17}{Sesion Tracking}
			
		\sse{Cookies}
			\stdImg{v1.18}{Arbeiten mit Cookies}
			\stdImg{v1.19}{Cookie empfangen}
			
		\sse{Request Dispatching (Redirecting)}
			\stdImg{v1.20}{}
			\expl{Forward \& include}{Bei forward komme ich nicht mehr zurück, beim Include bleibe ich beim aktuellen Request und binde andern Content ein.}	
			
			
		\sse{Request Filter}
			\stdImg{v1.21}{}
			\stdImg{v1.22}{}
			\stdImg{v1.23}{Verändern von Responses}
			\stdImg{v1.24}{Beispiel Responsewrapper}
			
\ch{Rest}
	\stdImg{v2.4}{}
	\stdImg{v2.3}{}
	\stdImg{v2.1}{}
	\stdImg{v2.2}{}
	
\ch{JSF}
	\stdImg{v3.1}{Ausführung von JSF}
	\stdImg{v3.2}{JSF Bestandteile}
	\expl{Java Beans}{Klasse die zu jedem Property public getter und setter besitzt}
	\stdImg{v3.3}{Web MVC}
	\stdImg{v3.4}{Architektur \& Request Handling}
	\stdImg{v3.5}{JSF Live Cycle}
	\expl{Komponenten Baum}{JSF baut aus den XML Templates Komponenten Bäume, die anschliessend gerendert werden. Platzhalter werden umgewandelt in die entsprechende Java Bean.}
	\expl{Validators}{Im Fehlerfall wird eine Fehlermeldung generiert, die im Faces Kontext gespeicher wird. Anschliessend wird direkt zur letzten Phase gesprungen.}
	
	
	
	
	
		
\ch{Clientseitige Technologien}
	
			






\end{document}
